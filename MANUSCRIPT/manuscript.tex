% !TeX program = pdfLaTeX
\documentclass[12pt]{article}
\usepackage{amsmath}
\usepackage{graphicx,psfrag,epsf}
\usepackage{enumerate}
\usepackage{natbib}
\usepackage{textcomp}
\usepackage[hyphens]{url} % not crucial - just used below for the URL
\usepackage{hyperref}
\providecommand{\tightlist}{%
  \setlength{\itemsep}{0pt}\setlength{\parskip}{0pt}}

%\pdfminorversion=4
% NOTE: To produce blinded version, replace "0" with "1" below.
\newcommand{\blind}{0}

% DON'T change margins - should be 1 inch all around.
\addtolength{\oddsidemargin}{-.5in}%
\addtolength{\evensidemargin}{-.5in}%
\addtolength{\textwidth}{1in}%
\addtolength{\textheight}{1.3in}%
\addtolength{\topmargin}{-.8in}%

%% load any required packages here



\usepackage{booktabs}
\usepackage{longtable}
\usepackage{array}
\usepackage{multirow}
\usepackage{wrapfig}
\usepackage{float}
\usepackage{colortbl}
\usepackage{pdflscape}
\usepackage{tabu}
\usepackage{threeparttable}
\usepackage{threeparttablex}
\usepackage[normalem]{ulem}
\usepackage{makecell}
\usepackage{xcolor}

\usepackage{booktabs}
\usepackage{longtable}
\usepackage{array}
\usepackage{multirow}
\usepackage{wrapfig}
\usepackage{float}
\usepackage{colortbl}
\usepackage{pdflscape}
\usepackage{tabu}
\usepackage{threeparttable}
\usepackage{threeparttablex}
\usepackage[normalem]{ulem}
\usepackage{makecell}

\begin{document}


\def\spacingset#1{\renewcommand{\baselinestretch}%
{#1}\small\normalsize} \spacingset{1}


%%%%%%%%%%%%%%%%%%%%%%%%%%%%%%%%%%%%%%%%%%%%%%%%%%%%%%%%%%%%%%%%%%%%%%%%%%%%%%

\if0\blind
{
  \title{\bf Effects without a Cause: The Search for Demographic Anomalies}

  \author{
        Mathew E. Hauer \thanks{Thanks y'all!} \\
    Department of Sociology, Florida State University\\
     and \\     Stephanie A. Bohon \\
    Department of Sociology, University of Tennessee - Knoxville\\
      }
  \maketitle
} \fi

\if1\blind
{
  \bigskip
  \bigskip
  \bigskip
  \begin{center}
    {\LARGE\bf Effects without a Cause: The Search for Demographic Anomalies}
  \end{center}
  \medskip
} \fi

\bigskip
\begin{abstract}
The proliferation of data, modern computing advances, and powerful
statistical algorithms unlock the potential search for hidden,
previously un- or understudied demographic anomalies. Here, we
investigate US state-level fertility and mortality time series since
1999 to uncover the hidden baby booms/busts and mortality
plagues?/non-plauges?. We find 38 states exhibited at least one
mortality anomaly, totalling more than 318k anomalous deaths and an
additional 6 states exhibited at least one mortality non-plague?
totalling 164k protective deaths. 12 states exhibited baby busts,
totalling more than 240k missing births and an additional 11 states
exhibited baby booms totalling more than 134k additional births. These
results suggest the widespread detection of demographic anomalies. Our
analysis does not examine the \emph{causes} of these anomalies and our
results point to important further research on the causes of anomalous
demographic behavior.\\
\end{abstract}

\noindent%
{\it Keywords:} mortality, fertility, causal inference
\vfill

\newpage
\spacingset{1.45} % DON'T change the spacing!

\newpage

\hypertarget{introduction}{%
\section{Introduction}\label{introduction}}

The proliferation of data, advances in high performance (``super'')
computing, and the development of powerful statistical algorithms mark
the era of ``Big Data'' or data science
\citep{van2016data, zikopoulos2011}, with the potential for researchers
to find hidden or understudied social phenomena
\citep{bohon2018demography}. While the availability of Big Data and high
performance computing allows novel exploration of data through causal
inference
\citep{bohon2018demography, rcausalimpact, shiffrin2016drawing},
relatively few studies utilize causal inference techniques in the study
of demographic phenomena. However, understanding social phenomena using
these advances reveals important insights into society
\citep{angrist1989lifetime, mas2009peers} and allows us to better
monitor population trends \citep{nobles2019, torche2015hidden}.

Causal inference approaches in social sciences have a long history
\citep{grimmer2015ppsp}. Often, the identification of a casual mechanism
requires either a randomized control trial (RCT) or expert knowledge
applied to a natural experiment. For population-level analysis,
expensive RCTs usually preclude their widespread adoption
\citep{west2008ajph}. Using natural experiments, however, provides
important and more widespread insights on the determinants of migration
after Hurricane Katrina in 2006
\citep{fussellRecoveryMigrationCity2014, horiDisplacementDynamicsSouthern2009},
mini baby booms after electrical blackouts \citep{fetzer2018jpe}, and
the highly publicized estimates of excess mortality after Hurricane
Maria in Puerto Rico in 2016
\citep{kishore2018mortality, santos2018use}. Both RCTs and natural
experiments follow traditional, hypothesis testing, inductive scientific
paradigms where a question is first posed and scientists generate or
find data to answer the question. Such approaches, while extensively
utilized and time-tested, are likely to miss important phenomena that
might go unnoticed.

Abductive modeling is the movement from the inductive to the deductive,
and sometimes back and forth, to reach conclusions
\citep{bryant2014realm} or ``inferring cause from effect''
\citep{Crowder2017}. In situations with large data sets, an abductive
approach is far superior to deductive hypothesis testing, as p-values
with an extremely large number of cases are far from revealing
\citep{head2015extent, nuzzo2014scientific} and we do not want to make
purely inductive inferences from data of questionable generalizability
\citep{ruggles2014big}. Owing to the long history of big datasets in
demographic research (one of the original ``big data'' resources
\citep{ruggles2014big}), the rich demographic data available in the
United States make the potential revelation of interesting and important
demographic phenomena not only possible, but extremely plausible -- even
if identification of the phenomena occurs without identifying the
underlying cause. In other sciences, the identification of effects
without causes has led to new theories of galactic migration of planets
and other breakthroughs \citep{gomes2005n}.

In this paper, we use modern statistical outlier detection algorithms
\citep{chen1993joint} on nearly twenty years of mortality and fertility
data at the US state-level to identify anomalous demographic behavior.
In essence, we identify \emph{effects without knowledge of the cause}.
We ask two questions regarding demographic anomalies: What are the
hidden baby booms/busts and mortality spikes/dips in the United States
over the last twenty years? We do not necessarily know the causes of
these anomalies but identifying them allows scientists with more
detailed knowledge of local population dynamics, state-level policy
making, or macro-economics to explain these phenomena post-hoc. From
explanations of phenomena that may have previously gone unnoticed,
demographers may be able to better forecast populations and provide
policy solutions for impending problems.

\hypertarget{materials-and-methods}{%
\section{Materials and Methods}\label{materials-and-methods}}

\hypertarget{method}{%
\subsection{Method}\label{method}}

We use a statistical time series outlier detection algorithm
\citep{chen1993joint}, implemented in the R programming language
\citep{rcore} via the tsoutliers package \citep{tsoutliers2019}. This
algorithm iteratively uses ARIMA models to 1) produce a counter-factual
time series to initially detect an outlier or anomaly, and 2) refit the
ARIMA with the outliers removed. Here we briefly summarize and describe
the method.

Often, the behavior of a time series can be described and summarized in
ARIMA models. If a series of values, \(y_t^*\), is subject to \(m\)
interventions or outliers at time points \(t_1,t_2,…,t_m\), then
\(y_t^*\) can be defined as

\[y_t^* = \sum_{j=1}^{m} \omega_jL_j(B)I_t(t_j) + \frac{\theta(B)}{\phi(B)\alpha(B)}\alpha_t\]

Where \(I_t(t_j)\) is an indicator variable with a value of 1 at
observation \(t_j\) and where the \(j\)th outlier arises, \(\phi(B)\) is
an autoregressive polynomial with all roots outside the unit circle,
\(\theta(B)\) is a moving average polynomial with all roots outside the
unit circle, and \(\alpha(B)\) is an autoregressive polynomial with all
roots on the unit circle.

We examine three types of outliers at time point \(t_m\): 1) additive
outliers (AO), defined as \(L_j(B)=1\); 2) level shift outliers (LS),
defined as \(L_j(B) = 1/(1-B)\); and 2) temporary change outliers (TC),
defined as \(L_j(B) = 1/(1-\delta B)\).

Colloquially, additive outliers arise when a single event causes the
time series to unexpectedly increase/decrease for a single time period;
level shift outliers arise when an event causes the time series to
unexpectedly increase/decrease for multiple time periods; and temporary
change outliers arise when an event causes the time series to
unexpectedly increase/decrease with lingering effects that decay over
multiple time periods.

An outlier is detected using a regression equation

\[ \pi(B)y_t^* \equiv \hat{e} = \sum_{j=1}^m \omega_j \pi(B)L_j(B)I_t(t_j) + \alpha_t \]

where \(\pi(B)=\sum_{i=o}^{inf} \pi_iB^i\). The identification of
outliers then involves a three step process to (1) identify all
potential outliers, \(t_j\) and \(L_j (B)\), (2) joint estimates of
model parameters and outlier effects are computed to identify
potentially spurious outliers, and (3) the outliers and effects are
re-estimated without spurious outliers. Of importance here is step 2,
which relies on some critical value above which an outlier at time point
m is considered spurious. Based on Chien and Liu's
\citeyearpar{chen1993joint} recommendation, we set a critical value of
3.5 which generally minimizes the possibility of Type I errors or false
positive outliers.

Outliers are then reported using a simple t-statistic.

\hypertarget{data}{%
\subsection{Data}\label{data}}

We search for demographic anomalies using the Center for Disease Control
and Prevention's online WONDER monthly fertility (2003-2017) and
mortality databases (1999-2016) for all fifty states and the District of
Columbia \citep{CDC_fert07, CDC_mort}. These data sets contain every
birth and death record in the United States over the time periods of
interest, representing the universe of both mortality and fertility data
in the US. These data are considered the ``gold standard'' of data
collections \citep{mahapatra2007civil} and have been considered
``complete'' since 1968 \citep{hetzel2016us}. We search over each state
equivalent's (n=51) mortality (n=228) and fertility (n=180) monthly time
series for a total of 20,808 state-months of data.

\hypertarget{results}{%
\section{Results}\label{results}}

We detect numerous anomalous mortality and fertility events at the US
state-level since 1999. A full listing of these anomalies can be found
in the \textbf{Supplementary Materials}. Here, we highlight highly
significant mortality and fertility events across all three types of
outliers (Additive Outliers, Level Shift Outliers, and Temporary Change
Outliers) where we have identified plausible explanations for these
outliers.

\hypertarget{mortality}{%
\subsection{Mortality}\label{mortality}}

We begin with mortality for New York State (Figure 1). We identify seven
anomalies in the mortality time series for New York, all with
t-statistics in excess of 3.91, making these anomalies highly
significant. The algorithm correctly identifies September 2001 as an
additive outlier (2001:09 t=6.40) where mortality in that month was
1,628 higher than anticipated. Clearly, this mortality event is likely
caused by the September 11 tragedy and the detection of this mortality
event provides confidence in our detection of other anomalies.

\newpage

\bibliographystyle{agsm}
\bibliography{mybibfile}

\end{document}
